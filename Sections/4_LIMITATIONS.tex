\let\xn\xnote
\section{Limitations}

A wise academic would be quick to acknowledge the many limitations that exist in conducting analyses of the kind borrowed by this study. One such limitation noted by Lynch is the important role discretion plays in affecting higher level outcomes, hidden behind the necessary decisions made over the course of the study.\xn{4-1} In the interests of complete transparency, three important considerations guide the path of this work.

First, in terms of time period, this study begins in 1998 as it coincides with the High Court adopting medium neutral citations (`MNC'), as endorsed by the Council of Chief Justices in May that year.\xn{4-2} By selecting a period covered entirely by MNCs, this study benefits from the inclusion of indirect citation counts, as explained in Part III B(1). However, excluded from this analysis is the broader trends in the 95 years of High Court operation prior to 1998.

Second, the choice of covariates was guided by identifying features of the collected data set that incorporate ``the run-of-the-mill business which constitutes the essence of the judicial process''.\xn{4-3} For instance, that the number of justices hearing a case varies is an indelible part of normal High Court procedure and provides a strong objective basis for drawing a limited set of conclusions. Accordingly, studies purporting to explain judicial behaviour in terms of background or presumptive ideology were not followed, as the demands of ``quantitative analysis ... seem to be fulfilled at too great a cost''.\xn{4-4}

Third, a handful of results are selected to undergo normalisation procedures, as is quite common in studies of this sort.\xn{4-5} Overall citations are normalised annually, by the number of judgments passed down in any given year. And, as already mentioned, per justice citation counts are normalised by the number of years a justice sits on the bench. Both strategies seek to control the effect of latent factors causing sporadic outliers---eg, the number of judgments written is likely to influence the number of available ``chances'' to cite a source.

There is also the matter of self and negative citations.\xn{4-6} None of the following results should be taken to have sanitised the effect of the possible bias attributable to these types of weighted citations. That being said, the damping factor leveraged in overall influence ratings is anticipated to minimise the effect of self-referential citations. With respect to the latter issue, previous work in this area has put cogent arguments against the need to disaggregate negative citations.\xn{4-7} If the Court deems it necessary to dismiss a source, its contents were likely nevertheless influential on developing the law, albeit in a different direction.\xn{4-8}

Additionally, quality control issues persist in automated data analyses, despite being efficient in principle. And although it would no doubt be preferable to manually verify the results alongside each parsed judgment, no such systematic review of the data was undertaken by this study. Any further work completed on this topic is encouraged to scrutinise ruthlessly.

Final mention should be made to the perennial issue of author name disambiguation.\xn{4-9} While attempts were made in Part III(A)(4) to disambiguate by surname, such efforts only solved half the problem. Ambiguity remained in both (i) different names used when publishing, and (ii) authors sharing the exact same name.\xn{4-10} Efforts were undertaken to manually reduce ambiguity to the greatest extent possible. However, recognition must be given to the possibility that individuals can change their name entirely, particularly by marriage.\xn{4-11} Future studies should consider the use of services providing unique identifiers such as ORCID\xn{4-12} or Google Scholar\xn{4-13} to automate an ambiguity reduction strategy.

