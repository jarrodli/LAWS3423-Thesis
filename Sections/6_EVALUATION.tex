\let\xn\xnote
\renewcommand\thesubsubsection{RQ\arabic{subsubsection}}

\newpage
\section{Evaluation}

\subsection{Are modern courts more open to citing journal articles?}

As a Court of final appeal, the High Court will frequently sit as arbiter for cases of great significance.\xn{6-1} Naturally, many of these cases raise difficult questions with few clear answers.\xn{6-2} Legal periodicals and treatises provide a wealth of knowledge, assisting the Court by provoking thoughtful discussion of ideas that might otherwise have been left overlooked. The following section thus closer examines the results outlined in Part V to determine whether the modern High Court is more receptive to citing journal articles.

Of the many trends elicited by this study, the recent decline in journal articles deserves a great deal of attention. The first 10 years of this study (1998-2006) comprise 46\% of the cases therefore cite more than 16 articles. The 2019-20 period accounts for a further 27\%. Half of the years examined by this study constitute 73\% of the top 15 cases citing the most journal citations. Therefore, the overall trend appears driven not by year, but by the type of case for which the Court passes judgment.

Take the decision in \emph{Love v Commonwealth}, with the most citations to journal articles, as an example.\xn{6-3} It was handed down in one of two years falling outside the overall negative trend. In that case, the Court held the term ``alien'' to be incapable of including Aboriginal and Torres Strait Islanders for the purposes of interpreting s 51(xix) of the Constitution.\xn{6-4} This decision garnered intense media attention and equal scrutiny from the other branches of government.\xn{6-5}

Similarly, the decision in \emph{Mann v Paterson Constructions Pty Ltd} further demonstrates that the type of case drives citations to academic articles.\xn{6-6} In \textit{Mann}, the Court was tasked with clarifying the remedies that are available to a party who partly performs a contract but subsequently terminates for repudiation.\xn{6-7} Here, foreclosing the route to restitution where an action for debt had accrued was a considerable departure from the prior legal position.\xn{6-8} 

Yet, despite the exclusion of \textit{Mann} from the weekly news cycle, both cases are widely considered to be important developments in the law. So far as a case is important, whether in a social or purely doctrinal sense, it is likely to grapple with challenging points of law, and attract citations to law journals.\xn{6-9} Such a conclusion follows from the results gathered over the 1998-2022 period.

The composition of the Court also impacts the overriding changes in citation practices over the 24 years of this investigation. As discussed, the opining justice bears significantly on the citation count.\xn{6-10} Such an observation implicitly recognises that periods of higher citation counts are more common with certain judges, including Courts composed with Kirby and Edelman JJ. Consistent with other longitudinal analyses of the High Court, associates are presumed to not affect citation rates.\xn{6-11}

% ADD a conclusion -- the court remains open to citing journal articles over the period under examination, but only where the nature of the case dictates its necessity.

\subsection{What influences the incidence of a citation to a journal article?}

Turning to the second question posed by this study, we explore a few reasons why journals are cited by the High Court. Few citation analyses examine causative relationships between citation incidence and explanatory variables, with none available for the High Court of Australia.\xn{6-12} Three covariates reinforce the findings of Section A, demonstrating that the relative importance of a case is largely determinative of whether a journal article is to be cited.

Difficult questions of law are likely to precipitate lengthy judgments. Indeed, so much was suggested by Smyth insofar as he claims that ``it is reasonable to believe that Justices will write longer judgments in cases which are legally more difficult or politically controversial or more likely to have a major social impact.''\xn{6-13} That the results show a statistically significant relationship between citation volume and number of paragraphs lends support to this proposition.

On questions concerning the constitutional powers of the Commonwealth, the \textit{Judiciary Act} contemplates the High Court will sit en banc.\xn{6-14} According to surveys of apex courts in other jurisdictions, an increased panel size in cases of great importance or difficulty is not uncommon.\xn{6-15} By revealing a strong relationship between panel size and journal citation incidence rates, the results continue to indicate that citation rates coincide with 

The final covariate dictates the relationship between journal citation volume and whether the judgment was written by a single author. Two reasons are put forward in an attempt to explain why justices writing alone cite journal articles more often than together. 

The first might be characterised broadly as the freedom of dissent. As noted by Kiefel CJ, ``[w]ith dissents, much greater liberty is possible and less care need be taken''.\xn{6-16} In support of this hypothesis, recent empirical evidence suggests that, more often than not, dissents are authored by a single justice.\xn{6-17} A second interrelated possibility is that single authored judgments simply do not face scrutiny by other judges and are, therefore, more resistant to competing conceptions of necessary legal analysis.

Other than the factors considered by the GLMM, two possible influences might be worthy of cursory discussion. First, the appellant's submissions in \textit{Love} made reference to a \textit{Cambridge Law Journal} article entitled ``The Correlation of Allegiance and Protection''.\xn{6-18} Notably, both the majority and minority refer to this journal article in defining the constitutional remit of alienage.\xn{6-19} 

Second, that many justices of the High Court were formerly silks and judges of intermediate appellate courts is trite.\xn{6-20} Among the 10 Australian authors with the highest \textit{j}-indexes, 4 sat as judges and 3 practised as barristers in Australia. The great number of former judges and barristers appearing in these results thus raises questions as to what articles High Court justices do in fact read.\xn{6-21}

While both these preliminary observations are sound in principle, both factors deserve justice in future regression analyses. Though, in any case, greater numbers of citations to journal articles are found in longer judgments, with larger panel sizes, and in judgments written by a single justice. Loosely, the incidence of journal citations might tentatively be stated to align with cases of general importance to the ``skeleton of the legal system''.\xn{6-22}

\subsection{Which academics influence the development of substantive law?}

The final question posed by this study can quickly be answered by examining at the final two tables in Appendix A.
