\let\xn\xnote
\section{Results}

The overarching purpose of this Part is to collate the results produced by applying the experimental design sketched out in Part III. In so doing, we briefly walk through the observed results, setting aside a more extensive analysis for Part VI. Three different sets of results are set out to provide an accurate account of the journal article citation environment. Though, while the results that follow are arranged to align closely with the research questions under examination, a holistic analysis will inevitably require support from all aspects of this study.

\subsection{Metrics}

Each of the following figures and tables presents the overall citation trends of the High Court of Australian between January 1998 and September 2022. We begin by examining the overall outgoing citation trend to journal articles. Figure 1 considers the number of times a judgment uniquely cites a journal article. These citations are ``unique'' because the collection methods outlined in Part III do not factor in subsequent abbreviated variants of the original fully formed citation.

Excepting 2019-20, figure 1 depicts a negative trend in the number of citations to journal articles over the period under examination. The \emph{mean} number of citations per year is $90.88$, with a \emph{standard deviation} of $43.32$. In terms of the ratio of citations per judgment, the local \emph{min} is $0.72$ (2015) and local \emph{max} is $2.52$ (1998). As outliers, both 2019 and 2020 vary significantly from the line of best fit reproduced below. Reasons for this are considered in the next Part.

\begin{figure}[!htpb]
    \centering
    \makebox[\textwidth][c]{\includesvg[width=1.2\textwidth,height=\textheight,keepaspectratio]{Sections/Figures/cites-overtime-v2.svg}}
    \caption{Total number of outgoing citations to journals cited by the High Court (normalised).}
\end{figure}

Perhaps of greater interest is the interplay between each sitting High Court justice and the number of times they cited a journal article. In view of the fact that the number of years a justice sits on the High Court varies wildly, figure 2 depicts the yearly average number of journal citations by each High Court justice. Kirby J is clearly the most amenable to citing journal articles (n=$82$), however, recognition should be given to the absence of judgments from justices appointed prior to 1998 from these results.\xn{5-1} Therefore, these results should not be taken to suggest the actual citation rates of a justice---they might be higher or lower depending on judgments as far back as 1987.

\begin{figure}[!htpb]
    \centering
    \includesvg[width=\textwidth,height=\textheight,keepaspectratio]{Sections/Figures/judges_normalised.svg}
    \caption{Aggregate citation count by justice of the High Court of Australia (normalised).}
\end{figure}

Many publishers, authors, and institutions often lay claim to the title of ``most cited by the High Court of Australia''. Figure 3 conclusively settles this debate for the modern High Court, ranking the top journals cited by the Court as assessed by the number of direct and indirect citations to a published article. To the dismay of many, these results do not support most of the conflicting claims to the top spot.\xn{5-2}

% TODO: Sheehy and Dumay Sydney Law Review is the most cited journal by the High Court of Australia, however, this is not so. see also ALJ ``current issues'' (1998) 72 ALJ 249

\begin{figure}[!htpb]
    \centering
    \makebox[\textwidth][c]{\includesvg[width=1.2\textwidth,height=\textheight,keepaspectratio]{Sections/Figures/journals.svg}}
    \caption{Top n journals cited by the High Court (n=$21$).}
\end{figure}

The \emph{Law Quarterly Review}\xn{5-3} is the most cited, with $192$ direct and $247$ indirect citations. At a close second, the \emph{Australian Law Journal}\xn{5-4} enjoys $172$ direct and $256$ indirect citations. Although the \emph{Journal of Contract Law}\xn{5-5} does not rank highly in terms of citations, it would be remiss to ignore the remarkable skew between the number of indirect (n=$43$) and direct (n=$9$) citations. Ultimately, these results are not novel and merely validate findings made by Smyth and Neilsen, who conducted a similarly spirited study.\xn{5-6}

For completeness, the most cited law journal articles are Gray's `Property in Thin Air' in the \textit{Cambridge Law Journal} and Salmond's `Citizenship and Allegiance' in the \textit{Law Quarterly Review} (n=$14$).\xn{5-7} Of the $1331$ cited articles, $885$ are only cited once and $1133$ are cited twice or less. Only $196$ articles are cited at least twice. The \emph{mean} number of citations is $7.29$, with a \emph{standard deviation} of $12.28$.

Table 1 reverses the investigation, listing the top cases ranked by number of citations to journal articles. The \emph{mean} number of citations per case with at least one outgoing citation is $3.94$, with a \emph{standard deviation} of $4.53$. 

Figure 4 continues the examination of journal articles, listing the authors most cited by the High Court. So far as raw numbers are concerned, Anthony Mason is the most cited author, with $132$ indirect and $58$ direct citations. George Williams follows closely behind with $60$ indirect and $53$ direct citations. The author trending highest in other decisions is Matthew Conaglen, who also happens to be the most cited author specialised in private law.

\begin{figure}[!htpb]
    \centering
    \makebox[\textwidth][c]{\includesvg[width=1.1\textwidth,height=\textheight,keepaspectratio]{Sections/Figures/authors.svg}}
    \caption{Top n journal authors cited by the High Court (n=$21$).}
\end{figure}

Comparatively, texts are cited heavily by the High Court, with a \emph{mean} of $48.37$ citations and a \emph{standard deviation} of $367.29$.\xn{5-8} By far, the most cited treatise is \textit{Cross on Evidence}, cited $2893$ times directly and $2592$ times indirectly.\xn{5-9} Maitland's  \textit{Equity} trails comfortably behind, with $675$ direct and $1180$ indirect citations.\xn{5-10} The only author to appear in the top $20$ cited journals and texts is Oliver W. Holmes. In respect of the latter, Holmes is cited exclusively for his seminal text \textit{The Common Law}.\xn{5-11}

Finally, Table 2 orders the citations by author, coalesces the article names into citation counts, and filters the results by \emph{j}-index. As depicted, few authors are able to achieve a \emph{j}-index greater than 2. And only Campbell managed to achieve a \emph{j}-index of $5$. Further, there are only $15$ authors with a \emph{j}-index greater than $2$ from a population size of $1105$. The table also shows that one out of three authors are, or were, judges.

\subsection{Causes}

Table 2 presents the estimated coefficients ($\beta$) for the Generalised Linear Mixed-effects Model defined in Part III. It contains six separate models, constituting different combinations of factors that are expected to drive the incidence of a citation. To provide a full picture, the intercept, or constant ($\beta_0$), estimated for each model is included in the table. Likewise, the log likelihood and Akaike Information Criterion (`AIC') are included to compare the ``fit'' between models.

Model 1 tests a basic assumption that the number of paragraphs in a judgment influences the number of citations to a secondary source. As this assumption meets the threshold test for statistical significance, it follows that every one unit increase in the number of paragraphs increases the expected log citation count by $0.003$. 

Model 2 introduces the ``lone opinion'' confounder. Also meeting the test for statistical significance, a coefficient of $1.445$ suggests that justices who write opinions alone cite journal articles more often when compared with justices in a joint opinion. 

Model 3 evaluates the impact of coram size and appeal status on the incidence of journal citations. Notably, a one unit increase in coram size corresponds to a log citation count increase of $0.164$, but the status of an appeal does not affect the citation incidence rate. 

Model 4 validates the assertions tested by Models 1 and 2.

\subsection{Rankings}

The final two tables annexed utilise the citation graph generated from the citation practices of judges and first-level academics. Adopting the simplest measure, Table 4 ranks journal authors by indegree citation count. That is, the number of inward connections between the ranked sink and all other sources. Of the 32 authors cited at least 20 times, the ratio of expertise between public and private law is $0.94:1$. In terms of author characteristics, 22 are domestic legal scholars, 4 are admitted as barristers in Australia, and 8 are, or were, Australian judges.

Using the random walk algorithm explained in Part III, Table 5 ranks authors by their influence on the development of substantive law. Zines tops the list, despite his absence from Table 4. Interestingly, only three out of the top ten influencers fit squarely within the public law sphere. Of the 50 ranked authors, 38 are domestic legal scholars, 6 are admitted as barristers in Australia, and 11 are, or were, Australian judges. 

Prior work by Smyth has produced rankings for journal authors as assessed from the point of view of journals alone.\xn{} Comparatively, the only authors to share in both Table 5 of this study (overall influence rankings) and Table 7 of Smyth's study (overall most prolific publishers in Australian law journals---including home review) are Michael Kirby, George Williams, Dan Meagher, Adrienne Stone, Roger Douglas, Enid Campbell, Ronald Sackville, Anthony Mason, and Mark Aronson.\xn{}
