\let\xn\xnote
\section{Conclusion and Future Work}

This study shows that the High Court has exhibited an open willingness to incorporate academic publications into judicial pronouncements. To the extent that the Court cites these journal articles, the data reveals it usually occurs in the presence of a larger coram with longer judgments that are written by a single judge. When combined with a comparative analysis between individual cases and years, it is argued that the High Court usually considers secondary material in cases of great social or doctrinal importance. Further, the annexed tables rank authors cited by the High Court in terms of their overall influence on legal development in Australia. It is demonstrated that the High Court tends to cite judges and barristers most frequently. Further, the results contradict claims that scientometric studies usually favour public law scholars over their private law counterparts.

Moving forward, a number of productive extensions may very well be explored in future scientometric studies. These suggestions can be divided into technical features and generic considerations. On the former, future work might follow prior scientometric studies that manually classify text according to polarity, however, leveraging Natural Language Inference models to automate this process in a scalable manner.\xn{7-1} Each citation might similarly be tagged with a topic to ascertain which areas of law are cited by the Court.\xn{7-2} Finally, running the current methodology over a longer time period and across multiple jurisdictions will ameliorate the types of claims that can be made about High Court citation practices since federation. More broadly, citations might be examined to gauge which authors might be ``monopolising'' the journal market.\xn{7-3} Additionally, ascertaining the treatment of a judgment is most desirable in understanding whether a cited passage subsequently forms part of formal legal principle rather than dissent.\xn{7-4} Another factor worthy of attention is the living status of authors. This can be factored into larger regression models with an aim to understand whether the ``better read when dead'' philosophy carries any weight in modern decisions.

%TC:ignore

% What about minority opinions. See also Dickenson's Arcade Pty Ltd v Tasmania [1974] HCA 9; 130 CLR 177 at 188. Constructing authority from fragmented authority. But also see Federation Insurance Ltd v Wasson [1987] HCA 34; 163 CLR 303 at [17]. Consideration of dissenting judgments. % + dissenting judgments

%TC:endignore