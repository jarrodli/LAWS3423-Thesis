\let\xn\xnote
\section{Introduction}

In times gone by, common law tradition had actively forestalled the possibility of vibrant collaboration between the judiciary and academia. Colloquially regarded as a `better read when dead' philosophy, legal scholarship was appreciated by courts only after the author passed away.\xn{1-1} While driven in part by convention, over the years many judges have directly questioned the practicality of the academy, contending that academics have a proclivity for writing in the abstract.\xn{1-2} Reflecting on this general stance, Chief Justice John Roberts of the United States Supreme Court remarked,

\begin{quote}
pick up a copy of any law review that you see, and the first article is likely to be, you know, the influence of Immanuel Kant on evidentiary approaches in 18\textsuperscript{th} Century Bulgaria, or something, which I’m sure was of great interest to the academic that wrote it, but isn’t of much help to the bar.\xn{1-3}
\end{quote}

Few judges in Australia have expressed similar concerns, instead opting to openly encourage written academic pieces on many aspects of Australian law.\xn{1-4} The immediate consequence of such an invitation is that, where the appropriate vehicle presents itself, courts will incorporate academic thought into legal reasoning.\xn{1-5} Though, quite apart from the extra-curial writing of judges, the High Court has, for some time, openly remarked upon the utility of secondary material in forming legal precedent. In \emph{Visy}, Kirby J turned to the use of academic literature in the routine exercise of construing ambiguous legislative provisions, noting that

\begin{quote}
[r]eliance upon the writings of legal scholarship, other disciplines (such as economic science in the context of the TPA), the law of other countries, or international law, represents an intellectual contribution to judicial reasoning and judgment. It is not normative. Nor is it prescriptive. But such learning can help a court better to understand the nature of the problem presented for its decision.\xn{1-6}
\end{quote}

On the face of it, the case of \emph{Stoddart} also opened the door to the consideration of secondary material in determining whether the common law recognised a form of spousal privilege.\xn{1-7} In dissent, Heydon J considered the purpose of secondary material at length, going so far as recognising, in narrow circumstances, jurist made common law where

\begin{quote}
[t]he evidence writers sought no glory ... in relation to the efforts of each other on spousal privilege. Their works reveal a general professional consensus. Writings of that kind generated out of that professional tradition are capable of constituting a source of law in their own right.\xn{1-8}
\end{quote}

Distinct from the adoption of scholarship in substantive law, Chief Justice Robert French recognised that empirical research on the operation of the law and the legal system creates opportunities to enliven interparty engagement within the legal community.\xn{1-9} Such a view values empirical studies attempting to understand the interaction between various actors within the legal system.\xn{1-10} Accordingly, this study seeks to lead with His Honour's proposal by examining the effect of journal articles on the development of legal principle.

In so doing, we will explore three key research questions based on data collected from High Court judgments between 1998 and 2022. They are, in turn,

\begin{enumerate}
    \item Has the practice of citing journal articles progressively changed?
    \item What influences the incidence of a citation to a journal article?
    \item Which academics influence the development of substantive law?
\end{enumerate}

Make no mistake, while the principal objective of this study is to answer questions concerning judicial behaviour, it does not attempt to do so in the abstract. Accordingly, Part II begins by detailing prior legal scientometric analyses of legal decisions, objections to the adoption of quantitative analyses in legal research, and institutional challenges these studies must face. Part III lays out the robust technical methods used to automate and process vast quantities of legal data. Part IV briefly considers any existing limitations. Part V presents the findings of this study. Part VI answers the initial research questions by evaluating the generated results. Part VII concludes and defines possible future avenues of  research.
