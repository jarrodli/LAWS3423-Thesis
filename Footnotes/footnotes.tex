%TC:ignore
% Don't include footnotes in word count.

% Footnotes.
% \fn{}{\iit{}.}

% INTRODUCTION
\fn{1-1}{Chief Justice Susan Kiefel, `The academy and the courts: What do they mean to each other today?' (2020) 44(1) \textit{Melbourne University Law Review} 447, 451. See also Neil Duxbury, `Better read when dead?' (2000) 32 \textit{Amicus Curiae} 25.}

\fn{1-2}{Richard A Posner, `The Judiciary and the Academy: A Fraught Relationship' (2010) 29(1) \textit{University of Queensland Law Journal} 13, 15-6.}

\fn{1-3}{`A Conversation with Chief Justice Roberts', \textit{Fourth Circuit Court of Appeals: 77th Annual Conference} (C-Span, 2011) 0:30:50-0:31:10 <\url{https://www.c-span.org/video/?300203-1/conversation-chief-justice-roberts}>.}

\fn{1-4}{See, eg, Kiefel (n 1) 455; Justice Michael Kirby, `Welcome to Law Reviews' (2002) 26(1) \textit{Melbourne University Law Review} 1, 6-11; Chief Justice Robert French, `Judges and Academics - Dialogue of the Hard of Hearing' (Speech, Australian Academy of Law, 30 October 2012) 14-5. But some academics find this direction concerning. See, eg, John Gava, `Law Reviews: Good for Judges, Bad for Law Schools?' (2002) 26(3) \textit{Melbourne University Law Review} 560, 575-6.}

\fn{1-5}{Sir Anthony Mason, `Future Directions in Australian Law' (1987) 13(3) \textit{Monash University Law Review} 149, 154; Justice Stephen Gageler, `The Coming of Age of Australian Law' in Barbara McDonald, Ben Chen and Jeffrey Gordon (eds), \textit{Dynamic and Principled} 8, 23-7}

\fn{1-6}{\textit{Visy Paper Pty Ltd v Australian Competition and Consumer Commission} (2003) 216 CLR 1, 25 [75] \sr{Visy}.}

\fn{1-7}{\textit{Australian Crime Commission v Stoddart} (2011) 244 CLR 554, 605-7 [132]-[135] \sr{Stoddart}.}

\fn{1-8}{Ibid 607 [135].}

\fn{1-9}{French (n 4) 9, 17-8.}

\fn{1-10}{Ibid.}

% SECTION 2
\fn{2-1}{See, eg, Brian T Detweiler, `May It Please the Court: A Longitudinal Study of Judicial Citation to Academic Legal Periodicals' (2020) 39(2) \textit{Legal Reference Services Quarterly} 87, 114-8. Detweiler uses a variety of wildcard searches in Lexis as a basic method for extracting journal articles from judgments. Joshua D Baker, `Relics or Relevant: The Value of the Modern Law Review' (2009) 111(3) \textit{West Virginia Law Review} 919, 928. Finding a noticeable decrease in the number of citations between 1971 to 1998. Baker also finds prestigious law journals to be cited more frequently. David L Schwartz and Lee Petherbridge, `The Use of Legal Scholarship by the Federal Courts of Appeals: An Empirical Study' (2011) 96(6) \textit{Cornell Law Review} 1345, 1361. Contrary to Baker, Schwartz and Lee found an increase in the number of citations to law journals between 1950 and 2008. This divergence in outcome might cause legal commentators who see these scientometric studies as unfit for purpose. Richard G Kopf, `Do Judges Read the Review - A Citation-Counting Study of the Nebraska Law Review and the Nebraska Supreme Court, 1972-1996' (1997) 76 \textit{Nebraska Law Review} 708, 717, 719-20. Former District Court Judge Kopf formulates a test for determining the qualitative impact of a citation by evaluating whether the relying paragraph directly mentions or follows the citation. On this method, see also Xiaodan Zhu et al, `Measuring academic influence: Not all citations are equal' (2014) 66(2) \textit{Journal of the Association for Information Science and Technology} 408, 413.}

\fn{2-2}{Kopf (n 11) 710. But see Lee Petherbridge and David L. Schwartz, `An Empirical Assessment of the Supreme Court's use of Legal Scholarship' (2012) 106(3) \textit{Northwestern University Law Review} 995, 1000-4. ``Content analysis'' refers to the practice of precisely determining the exact treatment of a journal citation through a process of manual review.}

\fn{2-3}{See generally Andrew Lynch, `Judicialization of Politics: The Interplay Of Institutional Structure, Legal Doctrine And Politics on the High Court of Australia', (2013) 34(2) \textit{Adelaide Law Review} 465, 466-7.}

\fn{2-3a}{See, eg, Brian Opeskin and Gabrielle Appleby, `Responsible Jurimetrics:
A Reply to Silbert’s Critique of the Victorian Court of Appeal' (2020) 94(12) \textit{Australian Law Journal} 923. Alan L. Tyree, `Fact Content Analysis of Case Law: Methods and Limitations' (1981) 22(1) \textit{Jurimetrics} 1, 2.}

\fn{2-3b}{See \textit{LOI n° 2019-222 du 23 mars 2019 de programmation 2018-2022 et de réforme pour la justice} (France) JO, 24 March 2019, art 33.}

\fn{2-3ba}{Russell Smyth, `Empirical studies of judicial behaviour and decision-making in Australian and New Zealand courts' in Nuno Garoupa, Lydia Tiede, and Rebecca D. Gill (eds), \textit{The Comparative Empirical Conundrum: Understanding High Court Behavior} (University of Virginia Press, 2020) 1, 13-4.}

\fn{2-3bb}{Federal Circuit and Family Court of Australia (Division 2), `Statement to Damien Carrick, ABC Law Report' (Media Release, 1 August 2022).}

\fn{2-3bc}{See, eg, Tonja Jacobi, Zoë Robinson and Patrick Leslie, `Querying the Gender Dynamics of Interruptions at Australian Oral Argument' (2020) 4 \textit{University of New South Wales Law Journal Forum} 1, 18-9. In response to an article by Amelia Loughland examining the influence gender carries on whether a High Court justice is likely to be interrupted during oral arguments, the authors carried out a regression analysis of the provided data and concluded that the results were of no statistical significance (ie, a \textit{p}-value $>$ 0.05).}

\fn{2-3c}{Australian Law Reform Commission, \textit{Judicial Impartiality} (Report No 138, December 2021) 495-6.}

\fn{2-4}{See, eg, Russell Smyth and Ingrid Nielsen, `The Citation Practices of the High Court of Australia, 1905-2015' (2019) 47(4) \textit{Federal Law Review} 655; Russell Smyth, `Other than ``Accepted Sources of Law''?: A Quantitative Study of Secondary Source Citations in the High Court' (1999) 22(1) \textit{University of New South Wales Law Journal} 19 \sr{Other than ``Accepted Sources of Law''?}; Russell Smyth, `What Do Trial Judges Cite? Evidence From the New South Wales District Court' (2018) 41(1) \textit{University of New South Wales Law Journal} 211 \sr{What Do Trial Judges Cite?}.}

\fn{2-5}{Smyth and Nielsen (n 20) 662-3.}

\fn{2-5a}{Ibid. See also Ingrid Nielsen and Russell Smyth, `One Hundred Years of Citation of Authority on the Supreme Court on New South Wales' (2008) 31(1) \textit{University of New South Wales Law Journal} 189, 195. The authors review trends over 100 years with data sampled from only 10 cumulative years.}

\fn{2-6}{G. S. Maddala, `Selectivity Problems in Longitudinal Data' (1978) 30/31 \textit{Annales de l'inséé} 423, 429. See especially from ``sample selectivity'' and truncation bias onwards.}

\fn{2-7}{See, eg, Smyth, \textit{What Do Trial Judges Cite?} (n 20) 229.}

\fn{2-8}{See especially Russell Smyth, ‘Who Publishes in Australia’s Top Law Journals’ (2012) 18(1) \textit{University of New South Wales Law Journal} 201, 219. The author concludes with: “[s]tudies of this sort could fruitfully be the subject of future research, where the focus of the research is on the influence ... on legal scholarship and reasoning in courts’ decisions in an Australian context”.}

\fn{2-9}{Lynch (n 13).}

\fn{2-10}{Lyria Bennett Moses, Nicola Gollan and Kieran Tranter, ‘The Productivity Commission:
a different engine for law reform?’ (2015) 24(4) \textit{Griffith Law Review} 657, 667-8.}

\fn{2-11}{For criticism on the use of empirical methods to determine academic impact, see Kathy Bowrey, ‘A Report into Methodologies Underpinning Australian Law Journal Rankings’ (Research Paper No 2016-30, University of New South Wales Faculty of Law, Prepared for the Council of Law Deans, 18 February 2016) 37-8.}

\fn{2-12}{Martin Szomszor, David A Pendlebury and Jonathan Adams, `How much is too much? The difference between research
influence and self-citation excess' (2020) 123(2) \textit{Scientometrics} 1119, 1120-1.}

\fn{2-12a}{Bowrey (n 28) 38.}

\fn{2-12b}{Ibid.}

\fn{2-13}{Jacobi, Robinson and Leslie (n 18) 13-5.}

\fn{2-14}{Carl T Bergstrom, Jevin D West and Marc A Wiseman, `The Eigenfactor™ Metrics' (2008) 28(45) \textit{Journal of Neuroscience} 11433.}

\fn{2-15}{See Lutz Bornmann and Hans-Dieter Daniel, `What Do We Know About the \textit{h} Index?' (2007) 58(9) \textit{Journal of the American Society for Information Science and Technology} 1, 1.}

% SECTION 3.1

\fn{3-1}{Robin Creyke, et al, \textit{Laying Down the Law} (LexisNexis Butterworths, 11\textsuperscript{th} ed, 2020) 228-30.}

\fn{3-2}{KR Chowdhary, \textit{Fundamentals of Artificial Intelligence} (Springer Nature, 2020) 604.}

\fn{3-3}{But see LT Olsson, \textit{Guide to Uniform Production of Judgments} (Australian Institute of Judicial Administration, 2\textsuperscript{nd} ed, 1999). Attention should be drawn to the divergence of citation formats from those outlined in uniform guides.}

\fn{3-4}{Melbourne University Law Review Association, \textit{Australian Guide to Legal Citation} (2020).}

\fn{3-5}{Ibid.}

\fn{3-6}{For an overview of regular expressions and the application of regular expressions to legal text more generally, see `Lexical Analysis Using Regular Expressions for Information Retrieval from a Legal Corpus' in Patricia Pesado and Gustavo Gil (eds), \textit{Computer Science – CACIC 2021} (Springer Nature, 2022) 312.}

\fn{3-7}{\textit{BarNet Jade} (Web Page, 16 October 2022) <\url{https://jade.io/}> \sr{Jade}.}

\fn{3-8}{\textit{BarNet Jade}, `Jade Browser' (Web Page, 16 October 2022) <\url{https://jade.io/search/collection=HCA:effective.since=883573200000:effective.until=1664114400000:order1.effectivedateasc=desc}>.}

\fn{3-9}{See, eg, Horacio Paggi et al, `Towards the definition of an information quality metric for information fusion models' (2021) 89 \textit{Computers and Electrical Engineering} 1.}

\fn{3-10}{Google, `General recommended practices for AI', \textit{Responsible AI practices} (Web Page, 16 October 2022) archived at <\url{https://archive.ph/czKPH}>.}

\fn{3-11}{Ibid.}

\fn{3-12}{Alireza Mansouri et al, `Named Entity Recognition Approaches' (2008) 8(2) \textit{International Journal of Computer Science and Network Security} 339.}

\fn{3-13}{See, eg, Gabriel Zingle et al, ‘Detecting Suggestions in Peer Assessments’ (Conference Paper, Proceedings of the International Conference on Educational Data Mining, 2 July 2019) 475.}
\fn{3-14}{See, eg, \textit{The Queen v Abdirahman-Khalif} (2020) 271 CLR 265, 295 [50]. In footnote 29, the Court references Lynch, McGarrity and Williams by surname.}

\fn{3-15}{For an example of data enrichment in practice, see Stainslaw AB Stane and Mariusz Zytniewsk, `Normative Multi-Agent Enriched Data Mining to Support E-Citizens' in Longbing Cao (ed), \textit{Data Mining and Multi-agent Integration} (Springer, 1\textsuperscript{st} ed, 2009) 321.}

\fn{3-16}{\textit{Australasian Legal Information Institute}, `LawCite Overview', \textit{What is LawCite?} (Web Page, 26 June 2014) archived at <\url{https://archive.ph/FFfop}>.}

\fn{3-17}{See, eg, \textit{Australasian Legal Information Institute}, `Formularism and Tort Law', \textit{Cases and Articles Cited} (Web Page, 20 October 2022) archived at <\url{https://archive.ph/tAv74}>.}

\fn{3-18}{Suzanne Scotchmer, `Standing on the Shoulders of Giants: Cumulative Research and the Patent Law' (1991) 5(1) \textit{Journal of Economic Perspectives} 29.}

\fn{3-19}{John PA Ioannidis, `Why Most Published Research Findings Are False' (2005) 2(8) \textit{PLoS Medicine} 696, 696, 699-700. See also Mark A Hall and Ronald F Wright, `Systematic Content Analysis of Judicial Opinions' (2008) 96(1) \textit{California Law Review} 63, 105-6.}

\fn{3-20}{Jason M Chin et al, `Improving the Credibility of Empirical Legal Research: Practical Suggestions for Researchers, Journals and Law Schools' (2021) 3(2) \textit{Law, Technology and Humans} 107, 108.}

\fn{3-21}{See, eg, \textit{Jade} (n 41). See also \textit{Australasian Legal Information Institute} (Web Page, 20 October 2022) <\url{https://www.austlii.edu.au/}>.}

\fn{3-22}{Chin et al (n 54) 119-20.}

\fn{3-23}{The dataset used in this study is publicly accessible, see \textbf{TODO: get link}.}

% SECTION 3.1

\fn{3b-0}{See Felicity Bell, `Empirical research in law' (2016) 25(2) \textit{Griffith Law Review} 262, 267.}

\fn{3b-4}{Norman Blaikie, \textit{Analysing Quantitative Data} (SAGE Publications, 2003) 39-40.}

\fn{3b-5}{Blaikie (n 59) 80.}

\fn{3b-5aa}{See \textit{Farah Constructions Pty Ltd v Say-Dee Pty Ltd} (2007) 230 CLR 89, 150 [134], affd \textit{Hill v Zuda Pty Ltd} (2022) 96 ALJR 540, 545 [25].}

\fn{3b-5a}{Bornmann and Daniel (n 34) 1.}

\fn{3b-5b}{John Clauser was awarded the 2022 Nobel Prize in Physics with a respectable but unextraordinary \textit{h}-index of 29. See, eg, Lutz Bornmann and Werner Marx, `The h index as a research performance indicator' (2011) 37(3) \textit{European Science Editing} 77, 77-8. But see JE Hirsch, `Does the h index have predictive power?' (2007) 104(49) \textit{Proceedings of the National Academy of Sciences of the United States of America} 19193, 19197.}

\fn{3b-5bb}{See Jingda Ding, Chao Liu and Goodluck Asobenie Kandonga, `Exploring the limitations of the h-index and h-type indexes in measuring the research performance of authors' (2020) 122(3) \textit{Scientometrics} 1303, 1305.}

\fn{3b-6}{P McCullagh and JA Nelder, \textit{Generalized Linear Models} (Chapman and Hall, 2\textsuperscript{nd} ed, 1989) 3-4.}

\fn{3b-7}{Ibid.}

\fn{3b-7a}{Judea Pearl, \textit{Causality} (Cambridge University Press, 2\textsuperscript{nd} ed, 2009) 134.}

\fn{3b-8}{Simon Rogers and Mark Girolami, \textit{A First Course in Machine Learning} (CRC Press, 2\textsuperscript{nd} ed, 2017) 41-3.}

\fn{3b-8a}{For an overview of data fitting practices including regularised least squares optimisation and maximum likelihood estimation, see Rogers and Girolami (n 68) 34, 70. See also Jared Wilber, `A Visual Introduction To (Almost) Everything You Should Know', \textit{Linear Regression} (Web Page, September 2022) archived at <\url{https://archive.ph/Oh5p2}>.}

\fn{3b-9}{Rogers and Girolami (n 68) 43.}

\fn{3b-10}{Ibid.}

\fn{3b-11}{See Sheldon M Ross, \textit{Introduction to Probability Models} (Academic Press, 12\textsuperscript{th} ed, 2019) 27.}

\fn{3b-12}{AJ Dobson and AG Barnett, \textit{An introduction to generalized linear models} (CRC Press, 3\textsuperscript{rd} edition, 2018) 197.}

\fn{3b-13}{Ibid. See also Joop J Hox, Mirjam Moerbeek and Rens van de Schoot, \textit{Multilevel Analysis: Techniques and Applications} (Taylor \& Francis Group, 2\textsuperscript{nd} ed, 2010) 151.}

\fn{3b-14}{Dobson and Barnett (n 73) 199.}

\fn{3b-15}{Mirko Bagaric and James McConvill, `The High Court and the Utility of Multiple
Judgments' (2005) 1(1) \textit{High Court Quarterly Review} 13, 21.}

\fn{3b-16}{The coefficients for the GLMM are estimated with random intercepts grouping data points for each justice (n=21) and year (n=24). For similar identification of non-independent variables, see Jacobi, Robinson and Leslie (n 18) 14. At footnote 51.}

\fn{3b-17}{Hox, Moerbeek and Schoot (n 78) 3-5. See also UCLA: Statistical Consulting Group, `Introduction to Generalized Linear Mixed Models', \textit{Statistical Methods and Data Analytics} (Web Page, April 2022) archived at <\url{https://archive.ph/a9T2s}>.}

\fn{3b-18}{Hox, Moerbeek and Schoot (n 78) 12.}

\fn{3b-19}{Hox, Moerbeek and Schoot (n 78) 13.}

\fn{3b-20}{Hox, Moerbeek and Schoot (n 78) 152. See also UCLA (n 78).}

\fn{3b-21}{Ronald L Wasserstein and Nicole A Lazar, `The ASA Statement on p-Values: Context, Process, and Purpose' (2016) 70(2) \textit{The American Statistician} 129, 131.}

\fn{3b-22}{Herman Aguinis, Matt Vassar and Cole Wayant, `On reporting and interpreting statistical significance and p values in medical research' (2019) 26(2) \textit{BMJ Evidence-Based Medicine} 39.}

\fn{3b-23}{See John PA Ioannidis et al, `A standardized citation metrics author database annotated for scientific field' (2019) 17(8) \textit{PLoS Biology} e3000384:1-6.}

\fn{3b-24}{See generally Avrim Blum, John Hopcroft and Ravindran Kannan, \textit{Foundations of Data Science} (Cambridge University Press, 2020) 97.}

\fn{3b-25}{Previous studies have found that judges on the High Court and Federal Court publish most actively in the \textit{Australian Law Journal}, see Russell Smyth, `Judges and Academic Scholarship: An Empirical Study of the Academic Publication Patterns Of Federal Court and High Court Judges' (2002) 2(2) \textit{Queensland University of Technology Law and Justice Journal} 198, 209.}

\fn{3b-26}{\textit{Forge v Australian Securities and Investments Commission} (2006) 228 CLR 45, 130 [216]. At footnote 271.}

\fn{3b-27}{Ernst Willheim, `Review of Australian Public Law Developments' (2006) 30(1) \textit{Melbourne University Law Review} 269.}

\fn{3b-28}{Ibid 278; Adrienne Stone, `The Limits of Constitutional Text and Structure: Standards of Review and the Freedom of Political Communication' (1999) 23(3) \textit{Melbourne University Law Review} 668.}

\fn{3b-28a}{\textit{Clubb v Edwards} (2019) 267 CLR 171, 332 [467]. At footnote 599.}

\fn{3b-29}{\textit{Brown v Tasmania} (2017) 261 CLR 328, 465 [430]. At footnote 454.}

\fn{3b-30}{Kurt Bryan and Tanya Leise, `The \$25,000,000,000 Eigenvector: The Linear Algebra behind Google' (2006) 48(3) \textit{SIAM Review} 569, 571.}

\fn{3b-31}{Blum, Hopcroft and Kannan (n 85) 97.}

\fn{3b-32}{Bryan and Leise (n 92) 571.}

\fn{3b-33}{Sergey Brin and Lawrence Page, `The anatomy of a large-scale hypertextual Web search engine' (1998) 30(1-7) \textit{Computer Networks and ISDN Systems} 109-10.}

\fn{3b-34}{Bryan and Leise (n 92) 580. By convention, $\upalpha$ is set to a value of $0.85$, see Brin and Page (n 95) 109.}

\fn{3b-35}{The mathematical basis for this algorithm lies in the formulation of a stochastic matrix that calculates the stationary probabilities for a random walker, see generally Gilbert Strang, \textit{Linear Algebra and Learning from Data} (Cambridge University Press, 2019) 312-18.}

\fn{3b-36}{The founders of Google describe the ability to teleport as a `damping factor', see Brin and Page (n 95) 109-10.}

\fn{3b-37}{Massimo Franceschet, `PageRank: Standing on the shoulders of giants' (2011) 54(6) \textit{Communications of the ACM} 92, 94.}

\fn{3b-38}{Ibid 95.}

\fn{3b-39}{Ibid.}

\fn{3b-40}{Bryan and Leise (n 92) 580.}

\fn{3b-41}{Ibid.}

\fn{3b-42}{Brin and Page (n 95) 109-10.}

\fn{3b-43}{Ibid. For a brief overview of the power method to compute the left eigenvector of large sparse matrices, see Mung Chiang, \textit{Networked Life: 20 Questions and Answers} (Cambridge University Press, 2012) 49-52.}


% SECTION 4

\fn{4-1}{Andrew Lynch, `The Gleeson Court on Constitutional Law: An
Empirical Analysis of Its First Five Years' (2003) 26(1) \textit{University of New South Wales Law Journal} 32, 35-6.}

\fn{4-2}{Joanne Sherman, `Council of Chief Justices Electronic Appeals Project —The Consultant’s Overview' (1998) 36 \textit{Computers \& Law} 29, 29-30.}

\fn{4-3}{Joel B Grossman, `Social Backgrounds and Judicial Decision-Making' (1966) 79(8) \textit{Harvard Law Review} 1551, 1563.}

\fn{4-4}{Ibid. See also Tonja Jacobi, Zoë Robinson and Patrick Leslie, `Comparative Exceptionalism? Strategy and Ideology in the High Court of Australia' \textit{American Journal of Comparative Law} (forthcoming), 27; Russell Smyth, `Explaining Voting Patterns on the Latham High Court 1935-50' (2002) 26(1) \textit{Melbourne University Law Review} 88, 103.}

\fn{4-5}{See generally Salvador García, Julián Luengo and Francisco Herrera, \textit{Data Preprocessing in Data Mining} (Springer, 2014) 46.}

\fn{4-6}{Fred R Shapiro, `The Most-Cited Law Review Articles' (1985) 73(5) \textit{California Law Review} 1540, 1543.}

\fn{4-7}{Ronen Perry, `The Relative Value of American Law Reviews: A Critical Appraisal of Ranking Methods' (2006) 11(1) \textit{Virginia Journal of Law and Technology} 1, 24.}

\fn{4-8}{Ibid; see also James Leonard, `Seein' the Cites: A Guided Tour of Citation Patterns in Recent American Law Review Articles' (1990) 34(2) \textit{Saint Louis University Law Journal} 181, 191.}

\fn{4-9}{Neil R Smalheiser and Vetle I Torvik, `Author Name Disambiguation' (2009) 43(1) \textit{Annual Review of Information Science and Technology} 6-1.}

\fn{4-10}{Ibid.}

\fn{4-11}{In 2016, Associate Professor Yvonne Corcoran-Nantes found more than 80\% of women take their husband's name after marriage. It is well-within contemplation that such practices disproportionately skew these types of longitudinal studies. See Sara Garcia, `Most Australian women still take husband's name after marriage, professor says', \textit{ABC News} (Web Page, 26 Apr 2016) archived at <\url{https://archive.ph/PxuKQ}>. See generally Anne F Young, Jennifer R Powers and Sandra L Bell, `Attrition in longitudinal studies: who do you lose?' (2006) 30(4) \textit{Australian and New Zealand Journal of Public Health} 353, 360.}

\fn{4-12}{ORCID, `About ORCID' (Web Page, 12 November 2022) archived at <\url{https://archive.ph/WYBDk}>.}

\fn{4-13}{Google, `About' \textit{Google Scholar} (Web Page, 12 November 2022) archived at <\url{https://archive.ph/rwv1N}>.}

% SECTION 5
\fn{5-1}{Including Toohey (1987), Gaudron (1987), McHugh (1989), Gummow (1995), and Kirby (1996) JJ.}

\fn{5-2}{See, eg, Benedict Sheehy and John Dumay, `Examining Legal Scholarship in Australia: A Case Study' (2021) 49(1) \textit{International Journal of Legal Information} 32, 34. see also Justice PW Young, ``Current issues'' (1998) 72(4) \textit{Australian Law Journal} 249, 254.}

\fn{5-3}{Ranked A* according to the Excellence in Research for Australia rankings (discontinued), see Australian Research Council, `2010 Final Journal List 100310 FOR WEBSITE', (Web Page, 19 November 2022) archived at <\url{https://archive.ph/WW12R}> (`ARC'). Ranked 44 in the United Kingdom with a combined score of (0.92) by the Washington and Lee University, see Washington and Lee University School of Law, `W\&L Law Journal Rankings', (Web Page, 19 November 2022) archived at <\url{https://archive.ph/D0SzE}>.}

\fn{5-4}{Ranked B according to the Excellence in Research for Australia rankings (discontinued), see ARC (n 119).}

\fn{5-5}{Ranked A according to the Excellence in Research for Australia rankings (discontinued), see ARC (n 119).}

\fn{5-6}{Smyth and Nielsen (n 20) 677-8.}

\fn{5-7}{Kevin Gray, `Property in Thin Air' (1991) 50(2) \textit{Cambridge Law Journal} 252; John W Salmond, `Citizenship and Allegiance' (1902) 18(1) \textit{Law Quarterly Review} 49.}

\fn{5-8}{See, eg, Smyth and Nielsen (n 20) 677.}

\fn{5-9}{JD Heydon, \textit{Cross on Evidence} (Lexis Nexis, 13\textsuperscript{th} ed, 2021).}

\fn{5-10}{FW Maitland, \textit{Equity} (Cambridge University Press, 1936).}

\fn{5-11}{OW Holmes Jr, \textit{The Common Law} (Little, Brown \& Company, 1881).}

\fn{5-12}{Smyth (n 25).}

\fn{5-13}{Ibid 241-2.}

% SECTION 6
\fn{6-1}{High Court of Australia, `Operation of the High Court', \textit{About} (Web Page, 13 November 2022) archived at <\url{https://archive.ph/LYu1x}>.}

\fn{6-2}{Graeme Orr, `Verbosity and richness: current trends in the craft of the High Court' (1998) 6(3)  \textit{Torts Law Journal} 291, 294.}

\fn{6-3}{(2020) 270 CLR 152, \sr{Love}.}

\fn{6-4}{Ibid 192 [81].}

\fn{6-5}{See Evan Young, ``A very bad thing': Peter Dutton slams High Court's Aboriginal `aliens' ruling' (Web Page, 13 February 2020) archived at <\url{https://archive.ph/9j87P}>. For an attempt by the government at overturning \textit{Love}, see Transcript of Proceedings, \textit{Montgomery v Minister for Immigration, Citizenship, Migrant Services and Multicultural Affairs} (High Court of Australia, Keane J, 29 November 2021).}

\fn{6-6}{(2019) 267 CLR 560 \sr{Mann}.}

\fn{6-7}{Ibid 618 [149].}

\fn{6-8}{\textit{Mann} (n 135) 645 [205].}

\fn{6-9}{Goh Yihan and Andrew Phang  `A Statistical Analysis of the Influence of the Journal of Contract Law in Commonwealth Court Decisions' 35(14) \textit{Journal of Contract Law} 14, 23.}

\fn{6-10}{For similar impact trends, see Smyth, \textit{Other than ``Accepted Sources of Law''?} (n 20) 36-7; Russell Smyth, `Academic Writing and the Courts: A Quantitative Study of the Influence of Legal and Non-Legal Periodicals in the High Court' (1998) 17(2) \textit{University of Tasmania Law Review} 164, 177.}

\fn{6-11}{Unlike the practice of assigning draft opinion writing to clerks in the United States Supreme Court, writing judgments tends not to be delegated to associates in the High Court, see Andrew Leigh, `Behind the Bench' (2000) 25(6) \textit{Alternative Law Journal} 295, 297. See also Andrew Lynch, `Individual Judicial Style and Institutional Norms' in Gabrielle Appleby and Andrew Lynch (eds) \textit{The Judge, the Judiciary and the Court} (Cambridge University Press, 2021) 214-5.}

\fn{6-12}{For example regression analyses of citations in New Zealand, see Russell Smyth, `Case complexity and citation to judicial authority: some empirical evidence from the New Zealand Court of Appeal' (2003) 10(1) \textit{Murdoch University Electronic Journal of Law}. But see Russell Smyth `Explaining historical dissent rates in the high court of Australia' (2003) 41(2) \textit{Commonwealth and Comparative Politics} 83.}

\fn{6-13}{Matthew Groves and Russell Smyth, `A Century of Judicial Style: Changing Patterns In Judgment Writing on the High Court 1903–2001' (2004) 32(2) \textit{Federal Law Review} 255, 262.}

\fn{6-14}{\textit{1903} (Cth) s 23(1) \sr{Judiciary Act}.}

\fn{6-15}{Benjamin Alarie, Andrew Green and Edward M Iacobucci, `Panel selection on high courts' (2015) 65(4) \textit{University of Toronto Law Journal} 335, 359.}

\fn{6-16}{Justice Susan Kiefel , `The individual judge' (2014) 88(8) \textit{Australian Law Journal} 554, 558.}

\fn{6-17}{Andrew Lynch and George Williams, `The High Court on Constitutional Law: The 2014 Statistics' (2015) 38(3) \textit{University of New South Wales Law Journal} 1078, 1086.}

\fn{6-18}{Commonwealth of Australia, `Defendant's Further Submissions', Submission in \textit{Love v Commonwealth}, No B43 Of 2018, 8 November 2019, 6.}

\fn{6-19}{\textit{Love} (n 132) 203 [107], 242 [248]. At footnotes 178 and 323.}

\fn{6-20}{George Winterton, `Appointment of Federal Judges in Australia' (1987) 16(2) \textit{Melbourne University Law Review} 185, 190.}

\fn{6-21}{See below Part VI(C).}

\fn{6-22}{Adopting the creative characterisation of the legal system by Brennan J, see \textit{Mabo [No 2]} (1992) 175 CLR 1, 29-30.}

\fn{6-23}{Bowrey (n 28) 55.}

\fn{6-24}{(2010) 38(3) \textit{Federal Law Review}.}

\fn{6-25}{William Gummow, `Foreword' (2010) 38(3) Federal Law Review 311 312.}

\fn{6-26}{Bowrey (n 28) 38.}

\fn{6-27}{\textit{Australian Constitution} s 75(v); \textit{Judiciary Act 1903} (Cth) s 30.}

\fn{6-28}{Justice Susan Kiefel, `On being a judge' (Speech, The Chinese University of Hong Kong, 15 January 2013) 1.}

\fn{6-29}{TODO}

\fn{6-30}{Despite this, Australian law journals are nevertheless understood to publish predominantly articles on issues concerning doctrinal law, see Michael Chesterman and David Weisbrot, `Legal Scholarship in Australia' (1987) 50(6) \textit{Modern Law Review} 709, 723.}

% CONCLUSION
\fn{7-1}{Pengfai Liu et al, `Pre-train, Prompt, and Predict: A Systematic Survey of Prompting Methods in Natural Language Processing' (2022) \textit{ACM Computing Surveys} (forthcoming).}
\fn{7-2}{David M Blei, Andrew Y Ng and Michael I Jordan, `Latent Dirichlet Allocation' (2003) 3 \textit{Journal of Machine Learning Research} 993; R Devika et al, `A Deep Learning Model Based on BERT and Sentence Transformer for Semantic Keyphrase Extraction on Big Social Data' (2021) 9 \textit{IEEE Access} 165252.}
\fn{7-3}{See generally Sara Reardon, ``Elite' researchers dominate citation space', \textit{Nature} (Web Page, 1 March 2021) <\url{https://doi.org/10.1038/d41586-021-00553-7}>.}
\fn{7-4}{See, eg, Bennett Moses, Gollan and Tranter (n 27).}

% APPENDIX A

\fn{a-1}{(n 132).}

\fn{a-2}{\textit{Regie Nationale Des Usines Renault SA v Zhang} (2002) 210 CLR 491.}

\fn{a-3}{(2003) 215 CLR 1.}

\fn{a-4}{(n 90).}

\fn{a-5}{(2011) 245 CLR 1.}

\fn{a-6}{(n 135).}

\fn{a-7}{(2012) 245 CLR 355.}

\fn{a-8}{(1999) 200 CLR 485.}

\fn{a-9}{(2021) 246 CLR 182.}

\fn{a-10}{(n 87).}

\fn{a-11}{(2020) 271 CLR 1.}

\fn{a-12}{(2001) 208 CLR 1.}

\fn{a-13}{(2007) 230 CLR 22.}

\fn{a-14}{(2003) 215 CLR 374.}

\fn{a-15}{(1998) 193 CLR 346.}

%TC:endignore
