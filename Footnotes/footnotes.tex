% Don't include footnotes in word count.
%TC:ignore

% Footnotes.
% \fn{}{\iit{}.}

% INTRODUCTION
\fn{1-1}{Test 1} 

% SECTION 2
\fn{2-1}{See, eg, Brian T. Detweiler, `May It Please the Court: A Longitudinal Study of Judicial Citation to Academic Legal Periodicals' (2020) 39(2) \textit{Legal Reference Services Quarterly} 87, 114-8. Detwiler uses a variety of wildcard searches in Lexis as a basic method for extracting journal articles from judgments. Joshua D. Baker, `Relics or Relevant: The Value of the Modern Law Review' (2009) 111(3) \textit{West Virginia Law Review} 919, 928. Finding a noticeable decrease in the number of citations between 1971 to 1998. Baker also finds prestigious law journals to be cited more frequently. David L. Schwartz and Lee Petherbridge, `The Use of Legal Scholarship by the Federal Courts of Appeals: An Empirical Study' (2011) 96(6) \textit{Cornell Law Review} 1345, 1361. Contrary to Baker, Schwartz and Lee found an increase in the number of citations to law journals between 1950 and 2008. This divergence in outcome lends credibility to legal commentators who see these scientometric studies as unfit for purpose. Richard G. Kopf, `Do Judges Read the Review - A Citation-Counting Study of the Nebraska Law Review and the Nebraska Supreme Court, 1972-1996' (1997) 76 \textit{Nebraska Law Review} 708, 717, 719-20. Former District Court Judge Kopf formulates a test for determining the qualitative impact of a citation by evaluating whether the relying paragraph directly mentions or follows the citation. On this method, see also Xiaodan Zhu et al., `Measuring academic influence: Not all citations are equal' (2014) 66(2) \textit{Journal of the Association for Information Science and Technology} 408, 413.}

\fn{2-2}{Kopf (n TODO) 710. But see Lee Petherbridge and David L. Schwartz, `An Empirical Assessment of the Supreme Court's use of Legal Scholarship' (2012) 106(3) \textit{Northwestern University Law Review} 995, 1000-4. ``Content analysis'' refers to the practice of precisely determining the exact treatment of a journal citation through a process of manual review.}

\fn{2-3}{See generally Andrew Lynch, `Judicialization of Politics: The Interplay Of Institutional Structure, Legal Doctrine And Politics on the High Court of Australia', (2013) 34(2) \textit{Adelaide Law Review} 465, 466-7.}

\fn{2-3a}{See, eg, Brian Opeskin and Gabrielle Appleby, `Responsible Jurimetrics:
A Reply to Silbert’s Critique of the Victorian Court of Appeal' (2020) 94(12) \textit{Australian Law Journal} 923. \textbf{TODO: get more egs including Jacobi + Lynch + Twomey.} Alan L. Tyree, `Fact Content Analysis of Case Law: Methods and Limitations' (1981) 22(1) \textit{Jurimetrics} 1, 2.}

\fn{2-3b}{See \textit{LOI n° 2019-222 du 23 mars 2019 de programmation 2018-2022 et de réforme pour la justice} (France) JO, 24 March 2019, art 33.}

\fn{2-3ba}{Russell Smyth, `Empirical studies of judicial behaviour and decision-making in Australian and New Zealand courts' in Nuno Garoupa, Lydia Tiede, and Rebecca D. Gill (eds), \textit{The Comparative Empirical Conundrum: Understanding High Court Behavior} (University of Virginia Press, 2020) 1, 13-4 \textbf{TODO: find offset from ch. start page}.}

\fn{2-3bb}{Federal Circuit and Family Court of Australia (Division 2), `Statement to Damien Carrick, ABC Law Report' (Media Release, 1 August 2022).}

\fn{2-3bc}{Jacobi, Robinson and Leslie (n TODO), 18-9. In response to an article by Amelia Loughland examining the influence gender carries on whether a High Court justice is likely to be interrupted during oral arguments, the authors carried out a regression analysis of the provided data and concluded that the results were of no statistical significance (i.e., a \textit{p}-value $<$ 0.05).}

\fn{2-3c}{Australian Law Reform Commission, \textit{Judicial Impartiality} (Report No 138, December 2021) 495-6.}

\fn{2-4}{See, eg, Russell Smyth and Ingrid Nielsen, `The citation practices of the High Court of Australia, 1905-2015' (2019) 47(4) \textit{Federal Law Review} 655; Russell Smyth, `Other than ``Accepted Sources of Law''?: A Quantitative Study of Secondary Source Citations in the High Court' (1999) 22(1) \textit{University of New South Wales Law Journal} 19; Russell Smyth, `What Do Trial Judges Cite? Evidence From the New South Wales District Court' (2018) 41(1) \textit{University of New South Wales Law Journal} 211.}

\fn{2-5}{Smyth and Nielsen (n TODO) 662-3.}

\fn{2-5a}{Ibid. See also Ingrid Nielsen and Russell Smyth, `One Hundred Years of Citation of Authority on the Supreme Court on New South Wales' (2008) 31(1) \textit{University of New South Wales Law Journal} 189, 195. The authors review trends over 100 years with data sampled from only 10 cumulative years.}

\fn{2-6}{G. S. Maddala, `Selectivity Problems in Longitudinal Data' (1978) 30/31 \textit{Annales de l'inséé} 423, 429. See especially from ``sample selectivity'' and truncation bias onwards.}

\fn{2-7}{Smyth (n TODO) 250. \textbf{TODO: change this, it doesn't reflect what is being said in the passage} Positing that more eminent international journals are not cited by lower courts because they ``focus on theoretical and policy developments in the law, which are of use to the appellate courts, but are not of particular relevance to a trial court.''}

\fn{2-8}{See especially Russell Smyth, ‘Who Publishes in Australia’s Top Law Journals’ (2012) 18(1) \textit{University of New South Wales Law Journal} 201, 219. The author concludes with: “[s]tudies of this sort could fruitfully be the subject of future research, where the focus of the research is on the influence ... on legal scholarship and reasoning in courts’ decisions in an Australian context”.}

\fn{2-9}{Lynch (n TODO).}

\fn{2-10}{Lyria Bennett Moses, Nicola Gollan and Kieran Tranter, ‘The Productivity Commission:
a different engine for law reform?’ (2015) 24(4) \textit{Griffith Law Review} 657, 667-8.}

\fn{2-11}{For criticism on the use of empirical methods to determine academic impact, see Kathy Bowrey, ‘A Report into Methodologies Underpinning Australian Law Journal Rankings’ (Research Paper No 2016-30, University of New South Wales Faculty of Law, Prepared for the Council of Law Deans, 18 February 2016) 37-8.}

\fn{2-12}{Ibid. \textbf{TODO: theres too many ibids here!!}}

\fn{2-12a}{Ibid.}

\fn{2-12b}{Ibid.}

\fn{2-13}{See, eg, Tonja Jacobi, Zoë Robinson and Patrick Leslie, `Querying the Gender Dynamics of Interruptions at Australian Oral Argument' (2020) 4 \textit{University of New South Wales Law Journal Forum} 1, 14-5.}

\fn{2-14}{Carl T. Bergstrom, Jevin D. West and Marc A. Wiseman, `The Eigenfactor™ Metrics' (2008) 28(45) \textit{The Journal of Neuroscience} 11433. \textbf{See generally TODO: Blum, Hopcroft, and Kannan book FOUNDATIONS OF DATA SCIENCE}}

\fn{2-15}{See Lutz Bornmann and Hans-Dieter Daniel, `What Do We Know About the \textit{h} Index?' (2007) 58(9) \textit{Journal of the American Society for Information Science and Technology} 1, 1.}

% SECTION 3

\fn{3-1}{Robin Creyke, et al., \textit{Laying Down the Law} (LexisNexis Butterworths, 11\textsuperscript{th} ed, 2020) 228-30.}

\fn{3-2}{K. R. Chowdhary, \textit{Fundamentals of Artificial Intelligence} (Springer Nature, 2020) 604.}

\fn{3-3}{But see L T Olsson, \textit{Guide to Uniform Production of Judgments} (Australian Institute of Judicial Administration, 2\textsuperscript{nd} ed, 1999). Attention should be drawn to the divergence of citation formats from those outlined in uniform guides.}

\fn{3-4}{Melbourne University Law Review Association, \textit{Australian Guide to Legal Citation} (2020).}

\fn{3-5}{Ibid. See also }

\fn{3-6}{For an overview of regular expressions and the application of regular expressions to legal text more generally, see `Lexical Analysis Using Regular Expressions for Information Retrieval from a Legal Corpus' in Patricia Pesado and Gustavo Gil  (eds), \textit{Computer Science – CACIC 2021} (Springer Nature, 2022) 312}

\fn{3-7}{\textit{Judgments \& Decisions Enhanced (JADE)} (Web Page, 16 October 2022) <\url{https://jade.io/}> \sr{JADE}.}

\fn{3-8}{\textit{Judgments \& Decisions Enhanced (JADE)}, `Jade Browser' (Web Page, 16 October 2022) <\url{https://jade.io/search/collection=HCA:effective.since=883573200000:effective.until=1664114400000:order1.effectivedateasc=desc}>.}

\fn{3-9}{See, eg, Horacio Paggi et al., `Towards the definition of an information quality metric for information fusion models' (2021) 89 \textit{Computers and Electrical Engineering} 1.}

\fn{3-10}{Google, `General recommended practices for AI', \textit{Responsible AI practices} (Web Page, 16 October 2022) archived at <\url{https://archive.ph/czKPH}>.}

\fn{3-11}{Ibid.}

\fn{3-12}{Alireza Mansouri et al., `Named Entity Recognition Approaches' (2008) 8(2) \textit{International Journal of Computer Science and Network Security} 339.}

\fn{3-13}{\textbf{TODO: find general convention for splitting dataset, else remove.}}
\fn{3-14}{See, eg, \textit{The Queen v Abdirahman-Khalif} (2020) 271 CLR 265, 295 [50]. In footnote 29, the Court references Lynch, McGarrity and Williams by surname.}

\fn{3-15}{For an example of data enrichment in practice, see Stainslaw A. B. Stane and Mariusz Zytniewsk, `Normative Multi-Agent Enriched Data Mining to Support E-Citizens' in Longbing Cao (ed), \textit{Data Mining and Multi-agent Integration} (Springer, 1\textsuperscript{st} ed, 2009) 321.}

\fn{3-16}{\textit{Australasian Legal Information Institute}, `LawCite Overview', \textit{What is LawCite?} (Web Page, 26 June 2014) archived at <\url{https://archive.ph/FFfop}>.}

\fn{3-17}{See, eg, \textit{Australasian Legal Information Institute}, `Formularism and Tort Law', \textit{Cases and Articles Cited} (Web Page, 20 October 2022) archived at <\url{https://archive.ph/tAv74}>.}

\fn{3-18}{Suzanne Scotchmer, `Standing on the Shoulders of Giants: Cumulative Research and the Patent Law' (1991) 5(1) \textit{Journal of Economic Perspectives} 29.}

\fn{3-19}{John P. A. Ioannidis, `Why Most Published Research Findings Are False' (2005) 2(8) \textit{PLoS Medicine} 696, 696, 699-700. See also Mark A. Hall and Ronald F. Wright, `Systematic Content Analysis of Judicial Opinions' (2008) 96(1) \textit{California Law Review} 63, 105-6.}

\fn{3-20}{Jason M. Chin et al., `Improving the Credibility of Empirical Legal Research: Practical Suggestions for Researchers, Journals and Law Schools' (2021) 3(2) \textit{Law, Technology and Humans} 107, 108.}

\fn{3-21}{See, eg, \textit{JADE} (n TODO). See also \textit{Australasian Legal Information Institute} (Web Page, 20 October 2022) <\url{http://www.austlii.edu.au/}>.}

\fn{3-22}{Chin et al. (n TODO) 119-20.}

\fn{3-23}{The dataset used in this study is publicly accesible, see \textbf{TODO: get link}.}

% SECTION 4

\fn{4-0}{See Felicity Bell, `Empirical research in law' (2016) 25(2) \textit{Griffith Law Review} 262, 267.}

\fn{4-4}{Norman Blaikie, \textit{Analysing Quantitative Data} (SAGE Publications, 2003) 39-40.}

\fn{4-5}{Blaikie (n TODO) 80.}

\fn{4-5a}{Bornmann and Daniel (n TODO) 1.}

\fn{4-5b}{John Clauser was awarded the 2022 Nobel Prize in Physics with a respectable but unextraordinary \textit{h}-index of 29. See, eg, Lutz Bornmann and Werner Marx, `The h index as a research performance indicator' (2011) 37(3) \textit{European Science Editing} 77, 77-8. But see J. E. Hirsch, `Does the h index have predictive power?' (2007) 104(49) \textit{Proceedings of the National Academy of Sciences of the United States of America} 19193, 19197.}

\fn{4-6}{P. McCullagh and J.A. Nelder, \textit{Generalized Linear Models} (Chapman and Hall, 2\textsuperscript{nd} ed, 1989) 3-4.}

\fn{4-7}{Ibid.}

\fn{4-7a}{Judea Pearl, \textit{Causality} (Cambridge University Press, 2\textsuperscript{nd} ed, 2009) 134.}

\fn{4-8}{Simon Rogers and Mark Girolami, \textit{A First Course in Machine Learning} (CRC Press, 2\textsuperscript{nd} ed, 2017) 41-3.}

\fn{4-8a}{For an overview of data fitting practices including regularised least squares optimisation and maximum likelihood estimation, see Rogers and Girolami (n TODO) 34, 70. See also Jared Wilber, `A Visual Introduction To (Almost) Everything You Should Know', \textit{Linear Regression} (Web Page, September 2022) archived at <\url{https://archive.ph/Oh5p2}>.}

\fn{4-9}{Rogers and Girolami (n TODO) 43.}

\fn{4-10}{Ibid.}

\fn{4-11}{}

\fn{4-12}{A. J. Dobson and A. G. Barnett, \textit{An introduction to generalized linear models} (CRC Press, 3\textsuperscript{rd} edition, 2018) 197.}

\fn{4-13}{Ibid. See also Joop J. Hox, Mirjam Moerbeek and Rens van de Schoot, \textit{Multilevel Analysis: Techniques and Applications} (Taylor \& Francis Group, 2\textsuperscript{nd} ed, 2010) 151.}

\fn{4-14}{Dobson and Barnett (n TODO) 199.}

\fn{4-15}{Mirko Bagaric and James McConvill, `The High Court and the Utility of Multiple
Judgments' (2005) 1(1) \textit{High Court Quarterly Review} 13, 21.}

\fn{4-16}{For similar identification of non-independent variables, see Jacobi, Robinson and Leslie (n TODO) 14. At footnote 51.}

\fn{4-17}{Hox, Moerbeek and Schoot (n TODO) 4-5. See also UCLA: Statistical Consulting Group, `Introduction to Generalized Linear Mixed Models', \textit{Statistical Methods and Data Analytics} (Web Page, April 2022) archived at <\url{https://archive.ph/a9T2s}>.}

\fn{4-18}{Hox, Moerbeek and Schoot (n TODO) 12.}

\fn{4-19}{Hox, Moerbeek and Schoot (n TODO) 13.}

\fn{4-20}{Hox, Moerbeek and Schoot (n TODO) 152. See also UCLA (n X).}

\fn{4-21}{Ronald L. Wasserstein and Nicole A. Lazar, `The ASA Statement on p-Values: Context, Process, and Purpose' (2016) 70(2) \textit{The American Statistician} 129, 131.}

\fn{4-22}{Herman Aguinis, Matt Vassar and Cole Wayant, `On reporting and interpreting statistical significance and p values in medical research' (2019) 26(2) \textit{BMJ Evidence-Based Medicine} 39.}

% SECTION 5
\fn{5-1}{Test 5}

% SECTION 6
\fn{6-1}{Test 6}

% CONCLUSION
\fn{7-1}{Test 7}

%TC:endignore
